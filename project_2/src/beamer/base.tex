\documentclass[xcolor=table]{beamer}
\usepackage[utf8]{inputenc}
\usepackage{amssymb}

\usetheme{Madrid}
\usecolortheme{default}

%------------------------------------------------------------
%This block of code defines the information to appear in the
%Title page
\title[Real Time Scheduling] %optional
{Project 2: Real Time Scheduling}

\author[ David Martínez, José Martínez] % (optional)
{David Martínez García\inst{1} \and José Martínez Hernández\inst{2}}

\institute[TEC] % (optional)
{
  \inst{1}%
  \url{david.martinez@estudiantec.cr}\\
  \textsc{2007058596}
  \and
  \inst{2}%
  \url{jpmh.1309@estudiantec.cr} \\
  \textsc{2020426476}
}

\date[August 2021] % (optional)
{MP-6117 Real Time Operating Systems, August 2021}

\logo{\includegraphics[height=0.4cm]{beamer/logo-tec.png}}

%End of title page configuration block
%------------------------------------------------------------

%------------------------------------------------------------
%The next block of commands puts the table of contents at the
%beginning of each section and highlights the current section:

\AtBeginSection[]
{
  \begin{frame}
    \frametitle{Table of Contents}
    \tableofcontents[currentsection]
  \end{frame}
}
%------------------------------------------------------------


\begin{document}

%The next statement creates the title page.
\frame{\titlepage}


%---------------------------------------------------------
%This block of code is for the table of contents after
%the title page
\begin{frame}
\frametitle{Table of Contents}
\tableofcontents
\end{frame}
%---------------------------------------------------------


\section{Algorithms}

%---------------------------------------------------------
\begin{frame}
\frametitle{Algorithms}
In this project, we will study the following real time scheduling algorithms:

\begin{itemize}
    \item Rate Monotonic (\textbf{RM})
    \item Earliest Deadline First (\textbf{EDF})
    \item Least Laxity First (\textbf{LLF})
\end{itemize}
\end{frame}

%---------------------------------------------------------

%---------------------------------------------------------
\begin{frame}
\frametitle{Rate Monotonic (\textbf{RM})}

\begin{alertblock}{General Description:}
Rate monotonic is a priority assignment algorithm used in real-time operating systems with a static-priority scheduling class. The static priorities are assigned according to the cycle duration of the job, so a shorter cycle duration results in a higher job priority.
\end{alertblock}

\end{frame}
%---------------------------------------------------------

%---------------------------------------------------------
\begin{frame}
\frametitle{Rate Monotonic (\textbf{RM})}

\begin{block}{Schedulability Test:}
\begin{equation}
\prod_{i=0}^{n} \left( \frac{E_i}{P_i}+1 \right) \leq 2
\end{equation}

\begin{itemize}
    \item $E_i$: execution time of the task $i$.
    \item $P_i$: period of the task $i$.
\end{itemize}
\end{block}

\end{frame}
%---------------------------------------------------------

%---------------------------------------------------------
\begin{frame}
\frametitle{Earliest Deadline First (\textbf{EDF})}

\begin{alertblock}{General Description:}
Earliest Deadline First is a dynamic priority scheduling algorithm used in real-time operating systems to place processes in a priority queue. Whenever a scheduling event occurs the queue will be searched for the process closest to its deadline. This process is the next to be scheduled for execution.
\end{alertblock}

\end{frame}
%---------------------------------------------------------

%---------------------------------------------------------
\begin{frame}
\frametitle{Earliest Deadline First (\textbf{EDF})}

\begin{block}{Schedulability Test:}
\begin{equation}
\sum_{i=0}^{n} \left( \frac{E_i}{P_i} \right) \leq 1
\end{equation}

\begin{itemize}
    \item $E_i$: execution time of the task $i$.
    \item $P_i$: period of the task $i$.
\end{itemize}
\end{block}

\end{frame}
%---------------------------------------------------------

%---------------------------------------------------------
\begin{frame}
\frametitle{Least Laxity First (\textbf{LLF})}

\begin{alertblock}{General Description:}
Least Laxity First  is a job level dynamic priority scheduling algorithm. It means that every instant is a scheduling event because laxity of each task changes on every instant of time. A task which has least laxity at an instant, it will have higher priority than others at this instant. Laxity is mathematically it is described as

\begin{equation}
    L_i = D_i - (t_i + C^r_i)
\end{equation}

\begin{itemize}
    \item $D_i$: next deadline of the task at $t_i$.
    \item $t_i$: current execution time.
    \item $C^r_i$: remaining computer time of the task at $t_i$.
\end{itemize}
\end{alertblock}

\end{frame}
%---------------------------------------------------------

%---------------------------------------------------------
\begin{frame}
\frametitle{Least Laxity First (\textbf{LLF})}

\begin{block}{Schedulability Test:}
TODO: Missing equations
\end{block}

\end{frame}
%---------------------------------------------------------
